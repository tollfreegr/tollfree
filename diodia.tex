\documentclass[a4paper,12pt]{article}
    \usepackage{fontspec}
    \usepackage{xunicode}
    \usepackage{xltxtra}
    \usepackage{xgreek}
    \setmainfont[Mapping=tex-text]{Iosevka Light}
    \usepackage{textcomp}
   	\usepackage{tikz}
   	\usepackage{graphicx}
	\usepackage[a4paper, total={15cm, 26cm}]{geometry}
	\renewcommand{\figurename}{\textbf{Εικόνα}}
	\usepackage[font={small,it}]{caption}
	\usepackage{float}
	\usepackage{hyperref}
	\parindent=0in
	\usepackage{ulem} %Πακέτο για strikethrough \sout{}
	

\title{\LARGE Αποφυγή Διοδίων για τη διαδρομή Αθήνα/Λαμία}
\date{}
\author{}

\begin{document}
\begin{titlepage}


\maketitle
\begin{figure}[hbp!]
	\centering
		\includegraphics[width=0.5\textwidth]{Nc-eu.svg.png}
\end{figure}
\abstract Σκοπός του παρόντος οδηγού πέρα από την εξοικονόμηση χρημάτων είναι και η ανάδειξη του παραλογισμού της λογικής, βάσει της οποίας τοποθετήθηκαν οι σταθμοί των διοδίων στην εθνική οδό. Διαβάζοντας θα ανακαλύψουμε τους τρόπους αποφυγής διοδίων στη διαδρομή Αθήνα/Λαμία καθώς και το συνολικό ποσό που δύναται να εξοικονομηθεί εφόσον ακολουθηθούν οι οδηγίες στο ακέραιο. Σε κάποιες περιπτώσεις λόγω κακής ποιότητας του οδοστρώματος της παράκαμψης ή έργων που ενδεχομένως να γίνονται στην περιοχή, λαμβάνοντας υπόψη και το ύψος του κομίστρου, επιλέγουμε να το καταβάλουμε. Η παράκαμψη φυσικά καταγράφεται και η επιλογή είναι του οδηγού αν θα την επιλέξει ή όχι. Προτρέπουμε την ελεύθερη διανομή αυτού του οδηγού ούτως ώστε να ωφεληθούν όσο το δυνατόν περισσότεροι οδηγοί ή ακόμη να προτείνουν βελτιώσεις\footnote{Για διορθώσεις ή βελτιώσεις στείλτε email στο \href{mailto:info@tollfreegr.com}{info@tollfreegr.com} και ενημερωθείτε για τις τελευταίες εκδόσεις του οδηγού στο \href{https://tollfreegr.com}{tollfreegr.com}}/διορθώσεις στα δεδομένα που ακολουθούν. 

\vspace{1.5cm}
\begin{center}
\begin{footnotesize}
{Αυτό το έργο χορηγείται με άδεια \href{http://creativecommons.org/licenses/by-nc-sa/4.0/}{Creative Commons Αναφορά Δημιουργού - Μη Εμπορική Χρήση - Παρόμοια Διανομή 4.0 Διεθνές}

\includegraphics[scale=0.5]{by-nc-sa.png}}
\end{footnotesize}
\end{center}

\thispagestyle{empty}
\end{titlepage}
\tableofcontents
\newpage
\begin{center}
\begin{huge}
Διαδρομή Αθήνα$\,\to\,$Λαμία
\end{huge}
\addcontentsline{toc}{section}{Διαδρομή Αθήνα προς Λαμία}
\section*{Αποφυγή διοδίων Αφιδνών}
\end{center}
\addcontentsline{toc}{subsection}{Αποφυγή διοδίων Αφιδνών}
Στην κατεύθυνση μας προς Λαμία στρίβουμε δεξιά στην ταμπέλα που γράφει Άγιος Στέφανος [\textbf{εικόνα 1}]. Συνεχίζουμε μέχρι τη διασταύρωση όπου στρίβουμε δεξιά όπως φαίνεται στην \textbf{εικόνα 2}.

\begin{figure}[hbp!]
	\centering
		\includegraphics[width=\textwidth]{images/athina-lamia/astefanos/astefanos0.png} 
			\caption{Στρίβουμε δεξιά στην έξοδο προς Α. Στέφανο}
		\includegraphics[width=\textwidth]{images/athina-lamia/astefanos/astefanos1.PNG}
			\caption{Στη διασταύρωση στρίβουμε δεξιά}
\end{figure}

Στη συνέχεια στο φανάρι στρίβουμε αριστερά [\textbf{εικόνα 3}], στην ανηφόρα και συνεχίζουμε ευθεία παράλληλα με την Εθνική οδό.

\begin{figure}[hbp!]
	\centering
		\includegraphics[width=\textwidth]{images/athina-lamia/astefanos/astefanos2.PNG}
			\caption{Μετά το φανάρι στρίβουμε αριστερά}
\end{figure}
\break
Συνεχίζουμε ευθεία μέχρι να περάσουμε αριστερά μας τα διόδια των Αφιδνών και στο τέλος του δρόμου φτάνουμε στη διασταύρωση όπου είναι τα ανθοπωλεία. Στρίβουμε αριστερά [\textbf{εικόνα 4}].

\begin{figure}[hbp!]
	\centering
		\includegraphics[width=\textwidth]{images/athina-lamia/astefanos/astefanos3.PNG}
			\caption{Στρίβουμε αριστερά}
\end{figure}
Συνεχίζουμε ευθεία μέχρι να φτάσουμε στη διασταύρωση της Μαλακάσας όπου και συνεχίζουμε ευθεία απέναντι (Ταμπέλα Αυλώνα 8) όπως και το φορτηγό βυτίο που φαίνεται στην \textbf{εικόνα 5}. \textbf{Προσοχή !!!}
Δεν πηγαίνουμε στην Αυλώνα αλλά προς την κατεύθυνση που πάει προς τα εκεί.  

\begin{figure}[hbp!]
	\centering
		\includegraphics[width=\textwidth]{images/athina-lamia/astefanos/astefanos4.PNG}
			\caption{Κατευθυνόμαστε ευθεία απέναντι}
\end{figure}
\break
Δε στρίβουμε πουθενά και συνεχίζουμε ευθεία. Θα περάσουμε την είσοδο του στρατοπέδου στο δεξί μας χέρι και θα συνεχίσουμε ευθεία για αρκετό δρόμο μέχρι να φτάσουμε να δούμε ένα βενζινάδικο της Shell  [\textbf{εικόνα 6}] στο δεξί μας χέρι που έχει και αέριο. 
\begin{figure}[hbp!]
	\centering
		\includegraphics[width=\textwidth]{images/athina-lamia/astefanos/astefanos5.PNG}
			\caption{Το βενζινάδικο της Shell. Φτάνουμε στο κομμάτι που πρέπει να στρίψουμε}
\end{figure}

\break
Σε μερικές εκατοντάδες μέτρα θα φτάσουμε στη διασταύρωση που δεξιά πάει για Χαλκίδα(ένας καλός τρόπος για αποφυγή και για όσους ενδιαφέρονται να πάνε Χαλκίδα)και αριστερά για Σχηματάρι. Στρίβουμε αριστερά  και αμέσως δεξιά όπως φαίνεται στις \textbf{εικόνες 7 και 8}.

\begin{figure}[hbp!]
	\centering
		\includegraphics[width=\textwidth]{images/athina-lamia/astefanos/astefanos6.PNG}
			\caption{Στρίβουμε αριστερά και ...}
\end{figure}
\begin{figure}[hbp!]
	\centering
		\includegraphics[width=\textwidth]{images/athina-lamia/astefanos/astefanos7.PNG}
			\caption{... αμέσως δεξιά στην ταμπέλα που λέει Λαμία}
\end{figure}

Βγαίνουμε στην Εθνική οδό.

\begin{center}
\section*{Αποφυγή διοδίων Θήβας}
\end{center}
\addcontentsline{toc}{subsection}{Αποφυγή διοδίων Θήβας}
Βρισκόμαστε στην ΕΟ και προχωράμε μέχρι να βρούμε την έξοδο για νοσοκομείο Θήβας [\textbf{εικόνα 9}].
\begin{figure}[hbp!]
	\centering
		\includegraphics[width=\textwidth]{images/athina-lamia/thiva/thiva1.PNG}
			\caption{Μόλις δούμε την ταμπέλα στην επόμενη έξοδο πάμε δεξιά}
\end{figure}
Αφότου μπούμε συνεχίζουμε ευθεία [\textbf{εικόνα 10}]
\begin{figure}[hbp!]
	\centering
		\includegraphics[width=\textwidth]{images/athina-lamia/thiva/thiva2.PNG}
			\caption{Πηγαίνουμε ευθεία απέναντι}
\end{figure}
\newpage
Καθ' όλη τη διάρκεια που θα είμαστε στον παράδρομο, θα έχουμε την Εθνική οδό στο αριστερό μας χέρι. Προορισμός μας είναι το καφέ 90, όπου από εκεί θα βγούμε ξανά στην Εθνική οδό. Στην πορεία θα συναντήσουμε 2 διασταυρώσεις όπου και στις 2 θα περάσουμε απέναντι [\textbf{εικόνες 11 -12}]. 
\begin{figure}[hbp!]
	\centering
		\includegraphics[width=\textwidth]{images/athina-lamia/thiva/thiva3.PNG}
			\caption{Πηγαίνουμε ευθεία απέναντι}
\end{figure}
\begin{figure}[hbp!]
	\centering
		\includegraphics[width=\textwidth]{images/athina-lamia/thiva/thiva4.PNG}
			\caption{Συνεχίζουμε ευθεία απέναντι}
\end{figure}
Από εκεί και μετά συνεχίζουμε ευθεία μέχρι να μπούμε στο χώρο στάθμευσης του καφέ 90, από όπου βγαίνουμε στην Εθνική Οδό.
\newpage
\begin{center}
\section*{Αποφυγή διοδίων Τραγάνας}
\end{center}
\addcontentsline{toc}{subsection}{Αποφυγή διοδίων Τραγάνας}
Έχουμε βγει από το parking του καφέ 90 και βρισκόμαστε στην Εθνική οδό με κατεύθυνση προς Λαμία. Αφού περάσουμε το Κάστρο Βοιωτίας αναζητούμε την ταμπέλα που αναγράφει "Θεολόγος - Μαλαισίνα" [\textbf{εικόνα 13}] και στην έξοδο στρίβουμε δεξιά [\textbf{εικόνα 14}].

\begin{figure}[hbp!]
	\centering
		\includegraphics[width=\textwidth]{images/athina-lamia/tragana/tragana1.PNG}
			\caption{Στην έξοδο στρίβουμε δεξιά}
	
\end{figure}
\begin{figure}[hbp!]
	\centering
		\includegraphics[width=\textwidth]{images/athina-lamia/tragana/tragana2.PNG}
			\caption{Στρίβουμε δεξιά (κατεύθυνση προς Προσκυνά)}
\end{figure}
\break			
Περνάμε κάτω από τη γέφυρα και κάνουμε αμέσως δεξιά [\textbf{εικόνα 15}]. Η ταμπέλα γράφει Προσκυνάς-Θεολόγος και στα 50 μέτρα έχουμε στο αριστερό μας χέρι, ένα βενζινάδικο της Elin.

\begin{figure}[hbp!]
	\centering
		\includegraphics[width=\textwidth]{images/athina-lamia/tragana/tragana3.PNG}
			\caption{Στρίβουμε δεξιά (Ταμπέλες Προσκυνά-Θεολόγο)}
\end{figure}
Το επόμενο κομμάτι έχει μια μικρή δυσκολία στον εντοπισμό του, ειδικά αν δεν έχουμε ξανακάνει το δρόμο. Μόλις φτάσουμε στο σημείο της \textbf{εικόνας 16} θα κάνουμε αριστερά. Μέχρι να φτάσουμε εκεί δεν έχει κάποια ταμπέλα για αυτό καλό είναι να μην αναπτύξουμε μεγάλη ταχύτητα. Στρίβουμε αριστερά.
\begin{figure}[hbp!]
	\centering
		\includegraphics[width=\textwidth]{images/athina-lamia/tragana/tragana4.PNG}
			\caption{Στρίβουμε αριστερά}
\end{figure}
\newpage
Συνεχίζουμε ευθεία και φτάνουμε στο χωριό Προσκυνάς το οποίο και διασχίζουμε. Μόλις βγούμε από το χωριό κάνουμε διαγώνια αριστερά [\textbf{εικόνας 17}].
\begin{figure}[hbp!]
%	\centering
		\includegraphics[width=\textwidth]{images/athina-lamia/tragana/tragana5.PNG}
			\caption{Στρίβουμε διαγώνια αριστερά}

%	\centering
		\includegraphics[width=\textwidth]{images/athina-lamia/tragana/tragana6.PNG}
			\caption{Στρίβουμε δεξιά (ταμπέλα προς Λαμία)}
\end{figure}			

Συνεχίζουμε ευθεία μέχρι να φτάσουμε στο τέλος του δρόμου, στη διασταύρωση της \textbf{εικόνας 18}. Στρίβουμε δεξιά (ταμπέλα προς Λαμία), περνάμε τη γέφυρα και αμέσως αριστερά [\textbf{εικόνα 19}]. Ακολουθούμε το δρόμο και βγαίνουμε στην Εθνική οδό.

\begin{figure}[H]
		\includegraphics[width=\textwidth]{images/athina-lamia/tragana/tragana7.PNG}
			\caption{Στρίβουμε αριστερά (ταμπέλα προς Λαμία)}
\end{figure}
\newpage
\begin{center}
\section*{Αποφυγή διοδίων Λαμίας\footnote{Λαμβάνοντας υπόψη την αναλογία ταλαιπωρίας/αξίας κομίστρου, επιλέγουμε να πληρώσουμε το κόμιστρο του συγκεκριμένου σταθμού}}
\end{center}
\addcontentsline{toc}{subsection}{Αποφυγή διοδίων Λαμίας}
Βρισκόμαστε στην Εθνική οδό και αφού περάσουμε τις σύραγγες στο ύψος του Αγίου Κωνσταντίνου μόλις δούμε την ταμπέλα για Καμμένα Βούρλα, κάνουμε δεξιά και στη διασταύρωση αριστερά [\textbf{εικόνες 20 - 21}]
		
\begin{figure}[H]
	\centering
		\includegraphics[width=\textwidth]{images/athina-lamia/lamia/lamia1.PNG}
			\caption{Μπαίνουμε στην έξοδο δεξιά}

	\centering
		\includegraphics[width=\textwidth]{images/athina-lamia/lamia/lamia2.PNG}
			\caption{Στρίβουμε αριστερά στην ΠΕΟ (Παλιά Εθνική Οδό)}
\end{figure}
Συνεχίζουμε μέχρις ότου να φτάσουμε στη διασταύρωση που φαίνεται στην \textbf{εικόνα 22}. Στρίβουμε δεξιά κάτω από τη γέφυρα και μπαίνουμε στον κυκλικό κόμβο όπου κατευθυνόμαστε προς την κατεύθυνση που λένε οι ταμπέλες "Camping Ε.Ο.Τ." [\textbf{εικόνα 23}]
\begin{figure}[H]
	\centering
		\includegraphics[width=\textwidth]{images/athina-lamia/lamia/lamia3.PNG}
			\caption{Κατευθυνόμαστε δεξιά κάτω από τη γέφυρα\newline}
	
	\centering
		\includegraphics[width=\textwidth]{images/athina-lamia/lamia/lamia4.PNG}
			\caption{Πάμε προς την κατεύθυνση των ταμπελών}
\end{figure}
Προχωράμε ευθεία μέχρι να περάσουμε στο αριστερό μας χέρι τα διόδια της Λαμίας (Αγία Τριάδα). Περνάμε κάτω από την Εθνική οδό και ως εκ τούτου την έχουμε στο δεξί μας χέρι. Συνεχίζουμε ευθεία και περνάμε πάλι κάτω από την ΕΟ. Θα φτάσουμε σε ένα τριγωνικό κόμβο όπου θα συνεχίσουμε διαγώνια δεξιά [\textbf{εικόνα 24}]. Συνεχίζουμε ευθεία μέχρι να βρεθούμε πίσω από τα Goody's της Λαμίας. Εκεί πηγαίνουμε διαγωνίως αριστερά όπως φαίνεται στην \textbf{εικόνα 25} και βγαίνουμε στην ΕΟ.
\begin{figure}[H]
	\centering
		\includegraphics[width=\textwidth]{images/athina-lamia/lamia/lamia5.PNG}
			\caption{Κατευθυνόμαστε δεξιά στον κόμβο}
	
	\centering
		\includegraphics[width=\textwidth]{images/athina-lamia/lamia/lamia6.PNG}
			\caption{Προχωράμε διαγώνια αριστερά}
\end{figure}

Είμαστε στην ΕΟ και σε λίγο φτάνουμε Λαμία. Αν συνεχίσουμε ευθεία για Βόλο,Λάρισα,Θεσσαλονίκη θα συναντήσουμε διόδια. Μπορούμε όμως στρίβοντας δεξιά να πάμε χωρίς επιβάρυνση προς Καρδίτσα,Τρίκαλα αλλά και Λάρισα καθώς και Βόλο (μέσω Φαρσάλων).
\vspace{12pt}
\paragraph{Μέχρι τη Λαμία εξοικονομήσαμε}
\begin{itemize}
\item Αυτοκίνητο = 12.60€
\item Μηχανή = 8.75€
\item Φορτηγό ή μικρό ρυμουλκούμενο = 31.55€
\item Νταλίκα = 44.2€
\end{itemize}
\newpage

\begin{center}
\begin{huge}
Διαδρομή Λαμία$\,\to\,$Αθήνα
\end{huge}
\addcontentsline{toc}{section}{Διαδρομή Λαμία προς Αθήνα}
\section*{Αποφυγή διοδίων Λαμίας\footnote{Λαμβάνοντας υπόψη την αναλογία ταλαιπωρίας/αξίας κομίστρου, επιλέγουμε να πληρώσουμε το κόμιστρο του συγκεκριμένου σταθμού. Στο συγκεκριμένο κομμάτι γίνονται έργα και ενδεχομένως κάποιοι δρόμοι να μην υφίστανται. Σε περίπτωση αλλαγών στο οδικό δίκτυο ή για τυχόν παραλείψεις δε φέρουμε ευθύνη.}}
\end{center}
\addcontentsline{toc}{subsection}{Αποφυγή διοδίων Λαμίας}
Ερχόμενοι από την Εθνική οδό Λαμίας - Λάρισας εισερχόμαστε στον Ε65 και συνεχίζουμε ευθεία μέχρι να φτάσουμε στη διχάλα της \textbf{εικόνας 26} όπου και κατευθυνόμαστε δεξιά.
 
\begin{figure}[H]
\includegraphics[width=\textwidth]{images/lamia-athina/lamia/lamia_002.jpg} 
\caption{Στρίβουμε δεξιά} 
\end{figure}

Μόλις φτάσουμε στη διασταύρωση της \textbf{εικόνας 27} (περίπου 200μ) περνάμε αριστερά και κάτω από το γεφυράκι και στη συνέχεια στρίβουμε δεξιά με κατεύθυνση προς Ανθήλη [\textbf{εικόνα 28}]. 
\begin{figure}[H]
\includegraphics[width=\textwidth]{images/lamia-athina/lamia/lamia_003.jpg}
\caption{Στρίβουμε αριστερά κάτω από τη γέφυρα}
\end{figure}
\begin{figure}[H]  
\includegraphics[width=\textwidth]{images/lamia-athina/lamia/lamia_004.jpg} 
\caption{Στρίβουμε δεξιά προς Ανθήλη} 
\end{figure}

Συνεχίζουμε για αρκετό κομμάτι του δρόμου ευθεία, περνάμε πίσω από τα Goodys και συνεχίζουμε μέχρι να φτάσουμε στο σημείο της \textbf{εικόνας 29} όπου και κάνουμε διαγώνια αριστερά. 
\begin{figure}[H]
\includegraphics[width=\textwidth]{images/lamia-athina/lamia/lamia_005.jpg}
\caption{Στρίβουμε διαγώνια αριστερά} 
\end{figure}

Συνεχίζουμε με κατεύθυνση προς Θερμοπύλες. Μόλις μπούμε στις Θερμοπύλες συνεχίζουμε ευθεία κάτω από τη γέφυρα και αμέσως δεξιά και στην πρώτη έξοδο δεξιά για να μπούμε στην Επαρχιακή οδό Αταλάντης-Εξάρχου. Συνεχίζουμε ευθεία μέχρι να βρεθούμε στον κυκλικό κόμβο των Καμμένων Βούρλων [\textbf{εικόνα 30}] και στη 2η έξοδο κατευθυνόμαστε δεξιά κάτω από τη γέφυρα και αμέσως αριστερά όπου και θα βρεθούμε στην ΠΕΟ (Παλιά Εθνική Οδό).
\begin{figure}[H]
\includegraphics[width=\textwidth]{images/lamia-athina/lamia/lamia_006.jpg}
\caption{Στρίβουμε δεξιά περνάμε κάτω από τη γέφυρα και αμέσως αριστερά}  
\end{figure}

Προχωρούμε ευθεία μέχρι το σημείο της διασταύρωσης της \textbf{εικόνας 31} (είναι το ίδιο που χρησιμοποιούμε και για κατεύθυνση προς Λαμία) όπου κάνουμε δεξιά και προχωρούμε περνώντας κάτω από τη γέφυρα. 
\begin{figure}[H]
\includegraphics[width=\textwidth]{images/lamia-athina/lamia/lamia_007.jpg}
\caption{Στρίβουμε δεξιά} 
\end{figure}


Βγαίνουμε στην Εθνική Οδό.

\newpage
\begin{center}
\section*{Αποφυγή διοδίων Τραγάνας}
\end{center}
\addcontentsline{toc}{subsection}{Αποφυγή διοδίων Τραγάνας}
Βρισκόμαστε στην ΕΟ. Αναζητούμε την ταμπέλα για Δελφούς-Αταλάντη \textbf{εικόνα 32} και στρίβουμε δεξιά. Στη διασταύρωση δεξιά [\textbf{εικόνα 33}] και αμέσως αριστερά [\textbf{εικόνα 34}]
\begin{figure}[H]
\includegraphics[width=\textwidth]{images/lamia-athina/tragana/tragana_008.jpg}
\caption{Στρίβουμε δεξιά}
\includegraphics[width=\textwidth]{images/lamia-athina/tragana/tragana_009.jpg} 
\caption{Στρίβουμε δεξιά...}
\end{figure}
\begin{figure}[H]
\includegraphics[width=\textwidth]{images/lamia-athina/tragana/tragana_010.jpg} 
\caption{...και αμέσως αριστερά}
\end{figure}
Συνεχίζουμε ευθεία μέχρι να περάσουμε στα αριστερά μας τα διόδια της Τραγάνας. Στο τέλος του δρόμου πηγαίνουμε δεξιά (ταμπέλα με κατεύθυνση Προσκυνά) [\textbf{εικόνα 35}]
\begin{figure}[H]
\includegraphics[width=\textwidth]{images/lamia-athina/tragana/tragana_011.jpg} 
\caption{στρίβουμε δεξιά}
\end{figure}
Μέσα στο χωριό στη διχάλα συνεχίζουμε αριστερά [\textbf{εικόνα 36}] (Ταμπέλα προς Εθνική Οδό)
\begin{figure}[H]
\includegraphics[width=\textwidth]{images/lamia-athina/tragana/tragana_012.jpg} 
\caption{στρίβουμε αριστερά}
\end{figure}
Συνεχίζουμε ευθεία μέχρι τη διασταύρωση όπου κάνουμε δεξιά προς ΕΟ [\textbf{εικόνα 37}]
\begin{figure}[H]
\includegraphics[width=\textwidth]{images/lamia-athina/tragana/tragana_013.jpg} 
\caption{στρίβουμε δεξιά}
\end{figure}
Τέλος φτάνουμε στη διασταύρωση [\textbf{εικόνα 38}]που βρίσκεται αμέσως μετά το βενζινάδικο της Elin. Εκεί συνεχίζουμε ευθεία απέναντι.
\begin{figure}[H]
\includegraphics[width=\textwidth]{images/lamia-athina/tragana/tragana_014.jpg} 
\caption{κατευθυνόμαστε ευθεία απέναντι}
\end{figure}
Βγαίνουμε στην Εθνική Οδό.
\newpage
\begin{center}
\section*{Αποφυγή διοδίων Θήβας (Δεν ισχύει)}
\end{center}
\addcontentsline{toc}{subsection}{Αποφυγή διοδίων Θήβας (Δεν ισχύει)}
\sout{Μπαίνουμε στο καφέ 90 και εισερχόμαστε στον παράδρομο που πάει ακριβώς παράλληλα με την ΕΟ. Περνάμε στο αριστερό μας χέρι τα διόδια της Θήβας και στη διασταύρωση κάνουμε αριστερά και ανεβαίνουμε στη γέφυρα [\textbf{εικόνα 39}]. }
\begin{figure}[H]
\includegraphics[width=\textwidth]{images/lamia-athina/thiva/thiva_015.jpg} 
\caption{στρίβουμε αριστερά πάνω στη γέφυρα}
\end{figure}
\sout{Αφότου περάσουμε απέναντι θα μας βγάλει στη διασταύρωση όπου δεξιά μας είναι το 90 (το απέναντι 90 από αυτό που μπήκαμε). Στρίβουμε αριστερά και συνεχίζουμε ευθεία. Θα περάσουμε 2 διασταυρώσεις όπου και στις 2 θα περάσουμε απέναντι} [\textbf{εικόνες 40 - 41}].
\begin{figure}[H]
\includegraphics[width=\textwidth]{images/lamia-athina/thiva/thiva_016.jpg} 
\caption{συνεχίζουμε ευθεία απέναντι}
\end{figure}
\begin{figure}[H]
\includegraphics[width=\textwidth]{images/lamia-athina/thiva/thiva_017.jpg} 
\caption{συνεχίζουμε ευθεία απέναντι (Ταμπέλα προς Αθήνα)}
\end{figure}
\sout{Στο τέλος του δρόμου στη διασταύρωση στρίβουμε δεξιά προς Αθήνα [\textbf{εικόνα 42}]}
\begin{figure}[H]
\includegraphics[width=\textwidth]{images/lamia-athina/thiva/thiva_018.jpg} 
\caption{στρίβουμε δεξιά}
\end{figure}
\sout{Περνάμε πάνω από τη γέφυρα και αμέσως παίρνουμε την έξοδο δεξιά όπως το βυτίο στην \textbf{εικόνα 43}}
\begin{figure}[H]
\includegraphics[width=\textwidth]{images/lamia-athina/thiva/thiva_019.jpg} 
\caption{στρίβουμε δεξιά}
\end{figure}
\sout{Βγαίνουμε στην Εθνική Οδό}
\newpage
\begin{center}
\section*{Αποφυγή διοδίων Αφιδνών}
\end{center}
\addcontentsline{toc}{subsection}{Αποφυγή διοδίων Αφιδνών}
Βρισκόμαστε στην Εθνική οδό. Αναζητούμε την έξοδο προς Οινόη και κάνουμε δεξιά [\textbf{εικόνα 44}]
\begin{figure}[H]
\includegraphics[width=\textwidth]{images/lamia-athina/afidnon/afidnon_020.jpg} 
\caption{στρίβουμε δεξιά}
\end{figure}
Αμέσως μετά στη διασταύρωση κάνουμε αριστερά [\textbf{εικόνα 45}]
\begin{figure}[H]
\includegraphics[width=\textwidth]{images/lamia-athina/afidnon/afidnon_021.jpg} 
\caption{στρίβουμε αριστερά}
\end{figure}
Συνεχίζουμε για ένα μεγάλο κομμάτι του δρόμου παράλληλα με την ΕΟ μέχρι να φτάσουμε στη διασταύρωση της εικόνας (είναι η πρώτη μεγάλη διασταύρωση που συναντάμε) όπου κάνουμε αριστερά [\textbf{εικόνα 46}] και αμέσως μετά μόλις περάσουμε κάτω από τη γέφυρα δεξιά [\textbf{εικόνα 47}].
\begin{figure}[H]
\includegraphics[width=\textwidth]{images/lamia-athina/afidnon/afidnon_022.jpg} 
\caption{στρίβουμε αριστερά...}
\includegraphics[width=\textwidth]{images/lamia-athina/afidnon/afidnon_023.jpg} 
\caption{... και αμέσως στρίβουμε δεξιά}
\end{figure}
Τώρα βρισκόμαστε στην ΠΕΟ Αθηνών Θεσσαλονίκης και συνεχίζουμε ευθεία μέχρι τη Μαλακάσα όπου και πάλι θα συνεχίσουμε ευθεία απέναντι όπως φαίνεται στην \textbf{εικόνα 48}.
\begin{figure}[H]
\includegraphics[width=\textwidth]{images/lamia-athina/afidnon/afidnon_024.jpg} 
\caption{συνεχίζουμε ευθεία απέναντι}
\end{figure}
Στην πορεία μας θα συναντήσουμε ακόμη μια διασταύρωση όπου και εκεί θα συνεχίσουμε απέναντι [\textbf{εικόνα 49}].
\begin{figure}[H]
\includegraphics[width=\textwidth]{images/lamia-athina/afidnon/afidnon_025.jpg} 
\caption{συνεχίζουμε ευθεία απέναντι}
\end{figure}
Το επόμενο σημείο προσοχής είναι τα ανθοπωλεία που τα συναντάμε όπως φαίνεται και στην εικόνα από τη διακριτική ταμπέλα "ΦΥΤΑ". Στρίβουμε δεξιά [\textbf{εικόνα 50}].
\begin{figure}[H]
\includegraphics[width=\textwidth]{images/lamia-athina/afidnon/afidnon_026.jpg} 
\caption{στρίβουμε δεξιά}
\end{figure}
Στην πορεία περνάμε τα διόδια των Αφιδνών στο δεξί μας χέρι. Συνεχίζουμε ευθεία μέχρι να φτάσουμε στον κόμβο του Αγ.Στέφανου όπου στο φανάρι κάνουμε δεξιά [\textbf{εικόνα 51}].
\begin{figure}[H]
\includegraphics[width=\textwidth]{images/lamia-athina/afidnon/afidnon_027.jpg} 
\caption{στρίβουμε δεξιά}
\end{figure}
Περνάμε κάτω από τη γέφυρα και αμέσως στρίβουμε δεξιά [\textbf{εικόνα 52}]. Στη μικρή διχάλα που ακολουθεί πάμε αριστερά.
\begin{figure}[H]
\includegraphics[width=\textwidth]{images/lamia-athina/afidnon/afidnon_028.jpg} 
\caption{στρίβουμε δεξιά}
\end{figure}
Βγαίνουμε στην Εθνική Οδό
\vspace{12pt}
\paragraph{Μέχρι την Αθήνα εξοικονομήσαμε}
\begin{itemize}
\item Αυτοκίνητο = 12.60€
\item Μηχανή = 8.75€
\item Φορτηγό ή μικρό ρυμουλκούμενο = 31.55€
\item Νταλίκα = 44.2€
\end{itemize}

\end{document}